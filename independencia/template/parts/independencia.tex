\chapter{La independencia}
\section{Como alcanzar la independencia}
Si has llegado hasta este punto probablemente te estes planteando esta cuestion. Y es que es cierto que no es algo trivial, yo \textbf{nunca te dare \textit{la solucion}, si no herramientas que considero te ayudaran a construir tu propia solucion}.

Esto para mi es una clave de la vida y no es que me desvie del tema, por que la \textbf{proactividad es un punto clave} si queremos ser independientes, si nos ponemos como meta la independencia tenemos que tener claro que tendremos que hacer mucho mucho trabajo que a lo mejor no es reconocido instantaneamente, pero ten por seguro que si has trabajado duro de forma honesta acorde a tus valores, los beneficios vendran por si solos.

Esto esta bien, es lo que hace la mayoria de la gente, mirar a hoy, al dia despues, a dentro de un rato, a una semana, a un mes...otro punto clave de la independencia considero que es \textbf{tener una vision de futuro clara} o al menos un camino, como comentaba en otro capitulo no lo tienes que tener todo clarisimo y fijo, pero si que conviene que definas tu camino al menos. Esta vision es lo que te va a ayudar a empujarte hacia ella y a hacer trabajo sin recompensas inmediatas, pues sabes que estas trabajando duro por algo mas alla de lo tangible, por un sentimiento, por una meta. Y diras, pero eso no me va a pagar las facturas, y yo te puedo decir que seguro que si lo hara, aunque te veas en la situacion de tener pocas posesiones materiales o problemas con el dinero, analizalo, pero si de verdad crees en tu vision, el dia llegara en que seras recompensado. Cuantas personas famososas incluso de la lista forbes no perdieron todo antes de llegar a donde estan o empezaron sin nada. Quiero hacer un pequeno apunte y es que cuando digo que definamos nuestro camino esto no es algo estatico, lo bonito de hecho es que mientras que lo vayamos recorriendo, probablemente lo iremos modificando poco a poco.

Intentare reflejar este conocimiento teorico con algunas anecdotas que relato a continuacion: En mi universidad un dia estaba hablando con una persona, me dijo que nunca podria llegar alto dentro de una empresa y mucho menos por mi mismo si no era por enchufe. Me parecio bastante curioso y se me quedo marcado, aunque yo creo que elijo mi propio destino y que puedo llegar a donde me proponga pero bueno. El caso es que en uno de los podcast de Fearless Motivation estaban comentando como gente que nace pobre se puede volver rica en terminos economicos por ejemplo, pero tambien en otros terminos como sociales, intelectuales...pero la parte importante venia cuando decian \textit{los padres ricos que nacieron en familias humildes les dan a sus hijos todo, todo menos las condiciones que les llevaron a ellos a ser ricos} a mi forma de ver con esto quieren decir que estos padres cuando eran ninos o adolescentes, sintieron una necesidad de trabajar mas, decidieron perseguir sus suenos y encontraron la manera, que ademas suele ser mas ingeniosa y sufrida en estos casos, de hacerlo sin el apoyo inmediato de sus familiares por que no tenian la capacidad economica para arrancar su empresa por ejemplo. Asi, ante la falta de algo como es el dinero, y el planteamiento de una meta o un sueno, consiguieron ese dinero, y ahora con el pagan todos los suenos de sus hijos que no se veran motivados a encontrar otra manera o trabajar duro por que con pedir el dinero a los padres lo pueden tener todo **(en algunos casos, notese que yo no tengo nada en contra de los padres ricos que se permiten lujos, es solo para exagerar el ejemplo).

Esta historia me parece paradojica pero curiosa, a parte me veo en la obligacion moral de comentar el techo que te estas poniendo a ti mismo si te levantas cada manana pensando que no puedes llegar mas alto que x posicion o no puedes llegar a tener mas de x dinero por el mero sitio en el que naciste. Es muy peligroso este sentimiento pero muy facil acoplarse a el pues asi puedes culpar a algo como el sistema de todas tus penurias (te justificas) lo cual es infinitamente mas sencillo que plantarte de cara al mundo y decir yo lo voy a conseguir, voy a trabajar duro y voy a hacer todo lo que este en mi mano por conseguir este sueno, voy a hacer sacrificios, no ire de fiesta, vivire con una politica economica de auesteridad y ahorro, no voy a caer en el consumismo o en otras trampas que me envuelven en una espiral de negatividad, voy a ser verdaderamente responsable y consecuente con mis acciones, si fracaso lo aceptare, aprendere la leccion y comenzare de nuevo pues ahora soy mas sabio y estoy un paso mas cerca de conseguir lo que me he propuesto, nada ni nadie tiene la fuerza psicologica para pararme los pies.



\section{Independencia en la toma de decisiones}
Ahora hablare de otro punto que considero fundamental, como te digo, todo esto son pequenas ideas para que mejores o evalues tu \textit{mindset}. \textbf{Tu eres el unico que toma tus decisiones}, si sientes que es la sociedad en general o un grupo de gente quien toma las decisiones por ti, probablemente estes dentro de una relacion de dependencia. Con esto no te digo que te vuelvas radical de la noche a la manana y digas que te vas a vivir solo, que te vas del trabajo, etc...\textit{(aunque esto puede puede llegar a ser algo positivo en algunos casos, no querria que se entendiera esa idea y no es un consejo que daria a alguien menos si no le conozco)}, lo que te digo es que evalues y que seas consciente de que aunque tu creas que haya cosas que tienes que hacer si o si por que otros te lo han dicho o "por que si", por que asi he nacido, por prejuicios sociales, etc...En ultima instancia tenemos que reconocer que los que tomamos las decisiones somos cada uno de nosotros, aunque esas decisiones sean seguir al rebano. 

En este ambito, nos tenemos que hacer fuertes, tenemos que comenzar a evaluar las areas en las que estamos siendo dependientes y comenzar a darnos cuenta de que las decisiones que tomamos no estan fundamentadas en ningun criterio que hayamos desarrollado nosotros. Hay que evaluar cada una de estas areas e ir haciendo cambios poco a poco, atreverte a decir no, a preguntar lo que nadie se atreve a preguntar, a cuestionar a los demas siempre con respeto, a plantear ideas que parezcan muy locas y que todo el mundo nos llame locos, que digan que no podemos conseguirlo. Con esto te animo a experimentar diferentes alternativas que te ayuden a recorrer tu camino, ver cual se adapta mejor, pero por favor, se consciente de que \textbf{tu has llegado hasta el sitio donde estas en la vida por tus propias decisiones, y si lo decides puedes cambiar las decisiones que vas a tomar y por tanto cambiar de sitio en la vida}, esto es lo que me parece fundamental, que nada ni nadie te pare.

Algo que me parece bastante curioso y me parece una de los efectos mas fascinantes de la vida es como tomamos decisiones en el momento presente, y que luego al evaluarlas desde un momento futuro nos damos cuenta de que podriamos haber tomado otras decisiones mejores. Esto me parece magia, pero a la vez fundamental, y es que repasar algunos eventos en los que tuvimos que decidirnos entre varias alternativas yo lo veo una cosa muy positiva pues podemos ver en el momento presente los diferentes escenarios en los que estariamos si nos hubieramos decantado por otra opcion. Esto es lo que creo que da \textbf{experiencia real} en la vida, ya que al darnos cuenta de que podriamos haber tomado alguna otra decision mejor y alguna otra decision peor, podemos ajustar los parametros para que la proxima vez que nos encontremos ante una situacion con caracteristicas similares, podamos testear la hipotesis que formamos la anterior vez y ver si se materializa.
\section{Independencia social}
Este es uno de los temas delicados si tenemos en cuenta los tiempos que corren, especialmente a la hora de escribir esto. A lo mejor soy solo yo el que piensa que la sociedad es generalmente dependiente unos de otros tanto a nivel de paises, organismos e individuos, pero creo que aunque eso fuera cierto, presentar esta idea puede ser interesante, sobre todo teniendo en cuenta que esto no va a ser una mera critica social si no que voy a presentar las alternativas en las que yo creo.

Para ilustrar este tema, voy a comentar el caso de un amigo que me mando un video, este consistia de un experimento muy famoso \textbf{INSERTAR AQUI} que consistia en pocas de palabras de ver la influencia que un grupo podia tener sobre un individuo a la hora de percibir la realidad. Habia varias lineas dibujadas en un papel, y varias personas tenian que decir cual era la mas corta. Lo que no se decia, era que todos los participantes eran actores menos uno. Las primeras veces se veia como la persona "real" decia lo que el pensaba por si mismo, es decir, actuaba de forma independiente, sin embargo, a medida que iban avanzando las rondas, iba dudando mas y mas y terminaba por decir lo mismo que los demas, pensando que su juicio era incorrecto al comienzo, y que sus propios ojos le enganaban a la hora de medir las lineas.

Esto me parece extremadamente curioso pues algo que es cuantificable y verificable matematicamente como es la longitud de una linea, y un algoritmo que seria muy sencillo para un ordenador como es calcular la linea de menor longitud, puede tornarse tan complejo en nuestra cabeza por el hecho de que estamos escuchando diferentes opiniones. No digo que tengamos que ser maquinas y que no nos debamos dejar influir por los demas en ciertos aspectos, solo que en aspectos tan fundamentales como estos debemos tener un minimo metodo cientifico con el que operar para ver si es cierto o no. En cuanto a este experimento, yo se que esta en nuestra naturaleza ser seres sociales que nos vemos en mayor o menor medida influenciados por lo que piensan los demas, pero en un caso real similar, tanto nos costaria sacar una regla y medir ? 

Entonces en que aspectos nos debemos "dejar influenciar"? 

Como ya he comentado cuando hablaba de los habitos, creo que es fundamental que a la hora de aprender algo nuevo, nos rodeemos de personas que lo potencien. Ahora bien, lo complejo aqui es elegir de que personas nos fiamos y de quienes no en cuanto a lo que nos van a ensenar, pues es posible que lo que nos digan se nos quede grabado si somos muy virgenes en el tema. Ante esta situacion, yo creo que lo mejor es contrastar opiniones, ver diferentes corrientes, por ejemplo si queremos aprender como funciona una sociedad a nivel economico, podemos tener en cuenta algunas opiniones de alguien que se declare comunista y de alguien que siga la escuela espanola o austriaca, os aseguro que las opiniones seran muy distintas. Asi, podremos 
\section{Independencia financiera}
\section{Independencia en la informatica}
