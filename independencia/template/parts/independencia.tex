\chapter{La independencia}
\section{Como alcanzar la independencia}
Si has llegado hasta este punto probablemente te estes planteando esta cuestion. Y es que es cierto que no es algo trivial, yo \textbf{nunca te dare \textit{la solucion}, si no herramientas que considero te ayudaran a construir tu propia solucion}.

Esto para mi es una clave de la vida y no es que me desvie del tema, por que la \textbf{proactividad es un punto clave} si queremos ser independientes, si nos ponemos como meta la independencia tenemos que tener claro que tendremos que hacer mucho mucho trabajo que a lo mejor no es reconocido instantaneamente, pero ten por seguro que si has trabajado duro de forma honesta acorde a tus valores, los beneficios vendran por si solos.

Esto esta bien, es lo que hace la mayoria de la gente, mirar a hoy, al dia despues, a dentro de un rato, a una semana, a un mes...otro punto clave de la independencia considero que es \textbf{tener una vision de futuro clara} o al menos un camino, como comentaba en otro capitulo no lo tienes que tener todo clarisimo y fijo, pero si que conviene que definas tu camino al menos. Esta vision es lo que te va a ayudar a empujarte hacia ella y a hacer trabajo sin recompensas inmediatas, pues sabes que estas trabajando duro por algo mas alla de lo tangible, por un sentimiento, por una meta. Y diras, pero eso no me va a pagar las facturas, y yo te puedo decir que seguro que si lo hara, aunque te veas en la situacion de tener pocas posesiones materiales o problemas con el dinero, analizalo, pero si de verdad crees en tu vision, el dia llegara en que seras recompensado. Cuantas personas famososas incluso de la lista forbes no perdieron todo antes de llegar a donde estan o empezaron sin nada. Quiero hacer un pequeno apunte y es que cuando digo que definamos nuestro camino esto no es algo estatico, lo bonito de hecho es que mientras que lo vayamos recorriendo, probablemente lo iremos modificando poco a poco.

Ahora hablare de otro punto que considero fundamental, como te digo, todo esto son pequenas ideas para que mejores o evalues tu \textit{mindset}. \textbf{Tu eres el unico que toma tus decisiones}, si sientes que es la sociedad en general o un grupo de gente quien toma las decisiones por ti, probablemente estes dentro de una relacion de dependencia. Con esto no te digo que te vuelvas radical de la noche a la manana y digas que te vas a vivir solo, que te vas del trabajo, etc...\textit{(aunque esto puede puede llegar a ser algo positivo en algunos casos, no querria que se entendiera esa idea y no es un consejo que daria a alguien menos si no le conozco)}, lo que te digo es que evalues y que seas consciente de que aunque tu creas que haya cosas que tienes que hacer si o si por que otros te lo han dicho o "por que si", por que asi he nacido, por prejuicios sociales, etc...En ultima instancia tenemos que reconocer que los que tomamos las decisiones somos cada uno de nosotros, aunque esas decisiones sean seguir al rebano. 

En este ambito, nos tenemos que hacer fuertes, tenemos que comenzar a evaluar las areas en las que estamos siendo dependientes y comenzar a darnos cuenta de que las decisiones que tomamos no estan fundamentadas en ningun criterio que hayamos desarrollado nosotros. Hay que evaluar cada una de estas areas e ir haciendo cambios poco a poco, atreverte a decir no, a preguntar lo que nadie se atreve a preguntar, a cuestionar a los demas siempre con respeto, a plantear ideas que parezcan muy locas y que todo el mundo nos llame locos, que digan que no podemos conseguirlo. Con esto te animo a experimentar diferentes alternativas que te ayuden a recorrer tu camino, ver cual se adapta mejor, pero por favor, se consciente de que \textbf{tu has llegado hasta el sitio donde estas en la vida por tus propias decisiones, y si lo decides puedes cambiar las decisiones que vas a tomar y por tanto cambiar de sitio en la vida}, esto es lo que me parece fundamental, que nada ni nadie te pare.


\section{Independencia en la toma de decisiones}
\section{Independencia social}
\section{Independencia financiera}
\section{Independencia en la informatica}
