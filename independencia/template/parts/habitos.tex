\chapter{El Habito}
\section{ Que es el habito }
Mi entendimiento de habito es una accion que repetimos en el tiempo \cite{lally2010habits} de tal forma que la hacemos nuestra en el sentido de que pasa a formar parte de nuestras actividades \underline{normales} y nos puede llegar a resultar fisica o psicologicamente dificil dejar de practicarlo. 
\section{ Que tiene que ver con la (in)dependencia?}
Que pasa cuando llega el verano y como estudiantes no tenemos que trabajar y por tanto nos podemos relajar y dormirmos a las tantas de la noche y levantarnos casi a la hora de comer?

Creamos ese habito y lo asociamos con sentimientos positivos como la felicidad, comodidad, descanso...etc. Esto es lo que en ocasiones provoca que cuando comienza el curso y yo este en la parada del autobus esperando para ir a la universidad, oiga comentarios como \textit{"que pereza"}, o \textit{"que conazo"}. Aqui hay un gran problema que es mas grave de lo que pueda parecer en un primer momento, y es que no solo hemos creado un habito que ya de por si podria resultarnos complicado o costoso renunciar a ello si no que lo hemos asociado a unos sentimientos positivos por lo que si tenemos la necesidad de romper este habito podriamos asociarlo a sentimientos negativos. 

Esto puede parecer una tonteria pero algo tan simple como levantarnos cada dia, si lo asociamos con sentimientos negativos (por que tenemos cierta dependencia de levantarnos mas tarde) puede ser la diferencia entre un dia productivo o miserable, y esto se puede extender en el tiempo.

Lo que quiero decir con esto no es que nos levantemos todos los dias a las 5 de la manana para aprovechar el dia y asi ser mas productivos. En epocas vacacionales yo personalmente y mcha otra gente recomienda descansar y disfrutar, ademas si gozamos del privilegio de no tener que trabajar o tener un horario muy flexible podemos aprovechar a dormir mas o ir a la cama mas tarde. Lo que yo vengo a reivindicar aqui realmente es que tenemos que ser conscientes de que cuando terminen las vacaciones y volvamos por ejemplo a clase es posible que nos tengamos que levantar a las 6 o 7.

\textit{Un gran poder conlleva una gran responsabilidad} y en este caso, el gran poder que tenemos es el de no tener que ajustarnos a un horario fijo (en un sistema en el que fueramos financieramente independientes quizas podriamos gozar de esto tambien) pero la responsabilidad es que llegado un punto tendremos que volver a hacerlo y tenemos que tener la fuerza interna, mental, suficiente para poder llevar a cabo esta accion.

El hecho de decir, vale, soy consciente de que estoy en un periodo vacacional y tengo menos obligaciones, ademas por mi situacion personal en los distintos ambitos fundamentales (salud, economia y relaciones) puedo permitirme relajarme durante este tiempo, pero soy perfectamente consciente de que no puedo establecer esto como mi normalidad a partir de ahora.

Si cometemos el error de establecer un habito como este como nuestra normalidad, cuando sabemos de antemano que por nuestras circunstacias ese habito solo lo podremos llevar a cabo durante un tiempo determinado (ej. 2 meses de verano) y luego tendremos que abandonarlo, estamos cometiendo un grave error. 

Mi consejo ante esta pequena anecdota es que disfrutemos de estos pequenos privilegios o lujos que podemos darnos pero no lo establezcamos como lo normal, que seamos conscientes de que dentro de un tiempo tendremos que dejar de hacer esto y que antes de permitirnos este lujo, evaluemos si somos suficientemente fuertes como para dejarlo.

Hay veces que creemos que somos fuertes para dejarlo y por tanto nos damos la licencia de permitirnos un lujo, pero a medida que va avanzando el tiempo y nos acostumbramos a ese lujo, la fuerza que teniamos o creiamos que teniamos al principio se va desvaneciendo.

Yo creo que cada uno nos tenemos que ganar los lujos por nosotros mismos, habiendo pasado por ganar la fuerza necesaria que corresponde con lo contrario a ese lujo. Asi, yo creo que en nuestro ejemplo anterior, si nos hemos levatado durante el curso a las 6 todos los dias religiosamente, \textbf{POR CONVICCION INTERNA NUESTRA Y NO POR IMPOSICION}, probablemente hayamos "ahorrado" la fuerza suficiente para poder permitirnos el lujo de dormir hasta tarde pero luego ser capaces de rapidamente "romper esa hucha" donde teniamos depositados los ahorros correspondientes a la fuerza para levantarnos temprano y asi adaptarnos a nuestra nueva realidad.

Es importante resaltar que he dicho que ahorramos fuerza durante un periodo en el que realizamos un habito de forma independiente, luego durante otra temporada nos damos el lujo de romper con el primer habito creando un segundo habito que es contrario al primero. Es importante decir que pasar de desperarse pronto a despertarse tarde es algo que no requiere mucha fuerza de voluntad o no requiere que "gastemos los ahorros de fuerza", al menos asi lo creo yo para la mayor parte de gente normal. Pero que pasa cuando tenemos que pasar de despertarnos tarde a desperatarnos temprano? Logicamente esto cuesta mas, requiere un esfuerzo adicional pues hemos estado tan comodos y hemos asociado sentimientos positivos a despertarnos tarde de tal forma que necesitamos "un empujon" para ir en contra de esos sentimientos positivos y de esa comodidad. 

Me gustaria hacer un inciso por que sin ser un profesional en esto veo que hay un cierto comportamiento que es mas comun de lo que parece, y es que cuando tenemos que romper con un habito positivo y comodo para nosotros hay gente que lo toma con una motivacion externa y esto lo considero un error. Me explico, por ejemplo para comenzar a ir a clase, y por tanto romper con los habitos que habiamos adoptado durante las vacaciones, podriamos coger motivacion externa de amigos que tambien van a clase y por tanto nos motivan a ir o de actividades que se realizan al comienzo de las clases para motivar a los estudiantes. Esto para mi es un error importante por que si no tenemos ahorros propios de fuerza interna para hacer algo si no que \textbf{dependemos} de los demas o de fuentes externas para que nos motiven a hacerlo, que pasara el dia que esas fuentes no esten? o si cambian las campanas y ya no nos motivan? El resultado de esto sera, en general, que no encontramos la motivacion para hacer las cosas. Por eso creo que el hecho de \textbf{depender} de fuentes externas para motivarnos es \textit{pan para hoy, hambre para manana}, poner tiritas en una herida profunda que requiere una operacion.

Esto viene a que debemos tener esos depositos de fuerza interna que sean en la medida de lo posible \textbf{independientes} de fuentes externas y por tanto cuando tengamos que levantarnos temprano y sintamos esa resistencia inicial, podamos echar mano de los ahorros de fuerza, que no es mas que la motivacion interna que tenemos para hacer algo. 

Todo esto anterior presuponia que sabemos que algo contrario a lo que nosotros nos hemos permitido como lujo va a ocurrir, pero que sucede cuando no sabemos de antemano que eso va a pasar? 

Este tipo de situaciones son lo que puede marcar la diferencia, ya que nosotros nos acostumbramos a un cierto estilo de vida a unos habitos que \textbf{dependen} de factores que aun siendo externos a nosotros, tenemos la conviccion interna de que esos factores no nos van a fallar, que siempre van a estar ahi cuando los necesitemos y la realidad es que no.

Hare un simil con un tema que probablemente pueda ser polemico pero es que a mi me parece que la gestion de las emociones de una persona es similar a la gestion economica, yo creo que al igual que en economia si queremos ser minimamente independientes debemos ahorrar dinero para cuando pueda venir una crisis, en la gestion de nuestras emociones debemos ahorrar en una forma menos tangible, deberiamos trabajar en nostros mismos cada dia un poco al igual que lo hacemos por dinero, por nosotros. Asi cuando venga una crisis emocional provocada por una situacion negativa podremos paliarla mejor.

Como podemos trabajar en nosotros mismos? Te estaras preguntando, pues es realmente sencillo para mi, yo creo que tenemos que hacer pequenas acciones que nos permitan definir quienes somos y seguir nuestro camino. Para acometer esto, nada mejor que tener muchas experiencias de gente sabia que ya recorrio el camino, por que al igual que en las finanzas es siempre mas conveniente aprender \textit{con el dinero de otros}, esto es, por sus experiencias y operaciones, yo creo que en el ambito sentimental y de la vida tambien se puede hacer esto.

Una de las formas que yo encuentro personalmente mas fascinante es escuchar historias de gente que estaba en situaciones malas y que salio adelante con mucha fuerza, asi me entreno en parte, sabiendo que por muy duro que sea lo que estoy pasando, puedo salir adelante y ademas salir reforzado. Esto lo he hecho durante muucho muucho tiempo, uno de los mejores medios es el canal de youtube de Fearless Motivation y similares.

Estas historias me han preparado para tratar situaciones muy dificiles que han terminado ocurriendo en mi vida, pero tambien el entrenamiento fisico es una de las actividades que a mi independientemente de todo lo demas me ha ayudado a salir adelante.

Puedo firmemente aconsejar estos medios para lograr ser una persona mas entrenada ante situaciones dificiles y saber salir de ellas de la mejor forma posible, incluso reforzados. 
\section{ Crear/cambiar habitos}

Como vemos esto de los habitos no es ninguna tonteria, un mal habito es como una mala inversion, y si lo asociamos a sentimientos muy positivos de forma que es muy dificil deshacernos de ello es como aplicar mucho apalancamiento en una operacion y que nos salga mal, nos podemos quedar estancados o irnos a la bancarrota.

Irse a la bancarrota en las finanzas o economia puede ser un golpe durisimo en nuestra vida, pero irse a la bancarrota internamente como personas, desarrollar y mantener habitos toxicos para nosotros mismos buscando el apoyo de los demas o el reconocimiento de factores externos puede determinar nuestra vida de una forma mas drastica de lo que la gente reconoce. 

Yo soy partidario como decia anteriormente de escuchar historias de las vidas de otras personas, especialmente de aquellos que son exitosos y a los que nos queremos parecer, y ver si forma de vida y habitos es, salvando las distancias, comparable a la nuestra y en que medida. Siempre podemos aprender algo y lo mejor, escuchar la historia de otras personas nos puede hacer reflexionar y llegar a la conclusion de que un habito de los que estabamos llevando a cabo podria ser un elemento contraproducente en el camino a completar nuestras metas.

Pero ojo y mucho cuidado, esto puede ser tanto positivo como negativo, por tanto yo creo que es determinante antes de "dejarnos influenciar" o tomar nota de lo que hacen otras personas deberiamos tener claro al menos cual es el camino que queremos seguir en cierta medida, no hace falta que hayamos trazado cada instante de nuestros proximos anos de vida y que lo sigamos a rajatabla, pero si que sepamos al menos el camino hacia el que queremos ir, que lo hayamos meditado y que creamos muy fuerte en nosotros mismos. Digo esto por que igual que yo me puedo plantar por ejemplo a ver un video de Steve Jobs, un documental sobre su vida por ejemplo, y captar como ideas positivas la forma en la que el creia en si mismo y lo claras que tenia las cosas, la vision que tenia y lo innovador que era para su tiempo, alguien puede ver esta historia desde otro punto de vista o le pueden contar otras partes y sacar como conclusiones que las drogas son positivas para tener ideas, que para poder perseguir tus suenos tienes que dejar la universidad, que acostarse con una persona y luego dejarla tirada incluso cuando tiene un hijo tuyo renegarlo...esto a mi personalmente no me representa segun mis valores.

Lo que quiero decir con esto es que cuidado, volviendo a nuestra historia del principio de fumar, existe el famoso argumento de \textit{pero si te vas a morir igual} que he escuchado muchas veces, y otro tipo de argumentos de este tipo que son mas bien creencias populares y que en ocasiones no estan bien fundamentados mas que en la opinion popular...pues resulta que este tipo de argumentos al menos en mi experiencia resultan muy cautivadores por el hecho de que es la creencia popular, es lo que la mayoria de personas piensa y por tanto pueden llegar a ser persuasivos. Es en estos momentos donde tenemos que plantearnos si este tipo de actos nos van a ayudar a avanzar en nuestro camino hacia nuestras metas y suenos o por el contrario nos van a hacer retroceder.

Hay un dicho en el mundo de la bolsa, que dice \textit{esto es 90\% psicologia y 10\% conocimiento}, pues yo creo que muchas de las situaciones de la vida, las decisiones que tomamos, se ven mucho mas influenciadas por la psicologia interna que por el conocimiento que tenemos.

Es por esto que yo considero fundamental tener una psicologia muy fuerte y que hayamos creado nosotros mismos.

Pero bueno, que nos desviamos del tema, esto simplemente eran unas notas que queria incluir sobre como adoptar nuevas ideas de habitos.

Yo recomiendo aprender una nueva cosa cada dia, un dia yo me puse a hacer ejercicio fisico y vi que no era malo en ello y he seguido el camino hasta un punto en el que llevo mucho tiempo en que el ejercicio fisico es un habito. Otro dia me sentia inspirado para tocar el piano y al ver que la curva de aprendizaje requeria mucho mas sacrificio del que yo estaba dispuesto a hacer, no me lo he tomado tan en serio, no es algo a lo que yo este dispuesto a dedicar el mismo esfuerzo que al entrenamiento fisico y eso no significa otra cosa que no lo quiero tanto. Un dia considere poner a prueba mi habilidad para montarme un ordenador de la forma mas barata posible, llevaba ya tiempo preparandome mentalmente para ello, habia creado ese habito, pero eso solo era la teoria, ahora me iba a probar a mi mismo en la parte practica.

Lo que quiero decir con esto es que es muy bueno probar cosas nuevas, por supuesto no todos los dias nos podemos plantear montar un pc, pero si que nos podemos plantear meternos en internet y buscar informacion sobre ello, estimar la cantidad de esfuerzo que requiere y decidir si nos merece la pena o no o si estamos dispuestos a hacerlo.

Me parece extraordinariamente importante recalcar que esta curiosidad por investigar nuevos temas ha de salir d e nosotros mismos, no de los demas, por ejemplo el gimnasio creo que es el area donde esto se ve mas claramente. A menudo vemos que hay gente que trata de ponerse en forma, se ponen unos objetivos y se apuntan a un gimnasio, pero en algunas ocasiones no lo estan haciendo por ellos mismos, si no por los demas: \textit{oye, tu tienes que ponerte en forma e} o \textit{despues de vacaciones nos apuntamos} pero solo lo hacen por la opinion que los demas tienen de ellos. Yo por ejemplo sin embargo comence a hacer ejercicio en mi casa tranquilamente, sin darle demasiada importancia, y asi segui y segui hasta que un dia me atrevi a ir a un parque de barras donde habia mas gente, pero como yo ya tenia el habito no me afectaba a niveles fundamentales la opinion que los demas tenian de mi, yo sabia quien era y lo que me habia costado llegar hasta ahi.

Lo que quiero decir con esto, es que los habitos mas potentes son aquellos que desarrollamos de forma independiente por curiosidad de descubrir una faceta de nosotros que nos gustaria explorar. En mi opinion los habitos que nos vienen impuestos por otras personas directamente o indirectamente por la opinion de la sociedad, no solo son debiles si no que pueden llegar al punto de debilitarnos a nosotros por que puede no gustarnos lo que hacemos y a partir de ahi podemos llegar a plantearnos cuestiones mas fundamentales sobre nuestro ser que nos pueden llevar a conclusiones muy buenas pero tambien a conclusiones muy negativas.

Abandonar un habito que tenemos puede llegar a ser complicado, pero mi mejor consejo para ello es quedarnos impactados ante la forma de vida que tienen otras personas a las que admiramos y que quizas no cuentan con ese habito, lo cual nos puede llevar a plantearnos si es realmente necesario para nosotros. Pero tambien es muy importante como comentaba anteriormente intentar minimizar el numero de habitos que tenemos que han sido impuestos por los demas, pues estos levantaran ,probablemente, dentro de nosotros sentimientos negativos. Si nos encontramos en la situacion de que ya estamos llevando a cabo habitos impuestos que no nos hacen sentir bien o nos hacen sentir mal, lo mejor que podemos hacer es buscar la manera de renunciar a ello para hacer algo con lo que nosotros mismos nos sintamos llenos y productivos.

